\documentclass{beamer}
\usepackage[utf8]{inputenc}
\usepackage[T1]{fontenc} 
\title{CS 581: Blockchain Science and Technology}
%\date[ISPN '80]{27th International Symposium of Prime Numbers}
\titlegraphic{\includegraphics[width=2cm]{logos/logo.pdf}}
%\author[Euclid]{Euclid of Alexandria \texttt{euclid@alexandria.edu}}
\usepackage[most]{tcolorbox}
\usepackage{multirow}
\usepackage{xcolor}
\usetheme{guwahati}

\begin{document}


\begin{frame}
  \titlepage
\end{frame}

\frame{
  \frametitle{Course Objectives}

\begin{tcolorbox}

  Bitcoin is the first \textcolor{red}{decentralized  cryptocurrency}. Nodes in the peer-to-peer bitcoin network verify transactions through cryptography and record them in a \textcolor{red}{public distributed ledger}, called a blockchain, \textcolor{red}{ without central oversight}

 \flushright \textit{Wikipedia}

 \end{tcolorbox}
\only<2->{
 Understand each of those ``features'', the tradeoffs involved and generalize when possible.}
}

\frame{
  \frametitle{Lecture Plan}
  
\begin{itemize}
 \item<1-> Currency/Money
 \item<2-> Permissionless Digital Identity
 \item<3-> Hashing
\end{itemize}

}

\frame{
  \frametitle{Currency}
  
\begin{itemize}
 \item<1->  Everyone deposits  their assets at a central godown.
 \item<2->  The godown issues certificates.
 \item<3->  People trade the certificates.
 \item<4->  Anyone can redeem the certificates for the underlying anytime.
 \item<5->  Standardize the certificates.
 \item<6->  Build electronic exchanges for trade.
\end{itemize}
}

\frame{
  \frametitle{Simplified Currency }
\begin{itemize}
 \item Just one type of asset!
 \item The certificate contains just a number.
\end{itemize}
  
}


\frame{
  \frametitle{Ledger for currency(First Attempt)}

  \begin{tcolorbox}[colback=white,colframe=black,width=0.8\textwidth]
    \begin{center}
      

\begin{tabular}{|c|c|c|}
\hline
\multicolumn{3}{|c|}{Universal Cash Positions}\tabularnewline
\hline
\hline
  \multicolumn{3}{|c|}{}\tabularnewline
  \multicolumn{3}{|c|}{}\tabularnewline
\hline
Name & Value & Remark\tabularnewline
\hline
 Aadarshraj& 35 & \tabularnewline
\hline
 Amjad&  87& \tabularnewline
\hline
 Gunjan&  -43& \tabularnewline
\hline
 Harsh& -34  & \tabularnewline
\hline
\end{tabular}
    \end{center}



    
  \end{tcolorbox}

}


\frame{
  \frametitle{Digital Identity via PKC}
\only<1->{
  \begin{tcolorbox}
    We can find $e$, $d$ and $n$ such that

    $$(m)^{ed} \equiv (m) \mod n $$
  \end{tcolorbox}
}



\only<2->{
   \begin{tcolorbox}
    Knowing $e$, and $n$ such that doesn't help in computing $d$.
  \end{tcolorbox}
}
}
\frame{
  \frametitle{RSA Cryptosystem}
  
\begin{itemize}
 \item Choose large primes $p$ and $q$. Let $n=pq$
 \item Choose large $d$ s.t $gcd(d, \phi(n)) = 1$.
 \item Fix $e$ as $d's$ inverse $mod \phi(n)$
 \end{itemize}

  \begin{tcolorbox}
  $(m)^{ed} = (m)^{k*\phi(n) +1}= (m^{\phi(n)})^{k}\times m = m \mod n$
  \end{tcolorbox}
 
}

\frame{
\frametitle{Immutability via Hashing}
}
\end{document}